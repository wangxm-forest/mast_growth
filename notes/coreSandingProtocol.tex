\documentclass[11pt,letter]{article}
\usepackage[top=1.00in, bottom=1.0in, left=1.1in, right=1.1in]{geometry}
\renewcommand{\baselinestretch}{1.5}
\usepackage{graphicx}
\usepackage{natbib}
\usepackage{amsmath}
\usepackage{hyperref}

\def\labelitemi{--}
\begin{document}

\title{Core sanding protocol} 
\date{\today}
\maketitle

\setlength{\parindent}{0pt}
\setlength{\parskip}{3pt}
\begin{enumerate}
\item Take a pair of cores from the box labeled "Cores need sanding."
\item Start with the 180/240 grit sandpaper on the orbital sander.
\item Set the speed to 2 or 3 depending on what feels comfortable. Place the orbital sander on the counter and try to keep it stable.
\item Slowly slide the core along the sander from both sides.
\item Progressively change to the next grit once the scratches from the previous grit have been removed.
\item Continue sanding up to 600 grit using the orbital sander.
\item Check the core under the microscope for any remaining scratches.
\item If no scratches are visible, place the two cores from the same individual together. Put them in the box labeled ``Cores ready for scanning''
\item There are some single cores, please send them as well and we will pair them up later.
\item Mark the end of each core with the color corresponding to the stand.
\item If scratches are still visible or the surface is unclear under the microscope, use the triangular sander to further polish with finer sandpaper until the surface is shiny and smooth.
\item Use compressed air to blow off dust from the sandpaper and the core after each sanding step.
\item At the end of the day, clean the bench and unplug the sanders. Ensure all cores are placed back in their designated boxes.
	\end{enumerate}
\end{document}