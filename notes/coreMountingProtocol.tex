\documentclass[11pt,letter]{article}
\usepackage[top=1.00in, bottom=1.0in, left=1.1in, right=1.1in]{geometry}
\renewcommand{\baselinestretch}{1.5}
\usepackage{graphicx}
\usepackage{natbib}
\usepackage{amsmath}
\usepackage{hyperref}

\def\labelitemi{--}
\begin{document}

\title{Core mounting protocol} 

\setlength{\parindent}{0pt}
\setlength{\parskip}{3pt}
\begin{enumerate}
\item Take a pair of cores from the paper bag.
\item Read the label on the straw.
\item Use scissors to open one or both ends of the straw, then use the stick to push the core out of the straw. \textbf{Be careful, the core might break into several pieces.}
\item Use a pencil to draw a line along the vertical vessels at both ends of the core.
\item Check if the core is twisted: If it is not twisted, go to step 7. If it is twisted, go to step 6.
\item If the core is twisted, put it back into the straw, use tape or staples to close both ends, and tape both cores together. Put them into the yellow basket.
\item Take two mounts and place them side by side. Write down the information on the label on both mounts. If the label is handwritten and incomplete, the format should be: MORA 2024 StandID Species Tag (1/2) barkside.
\item If the core is broken, use wood glue (the yellow one) to glue the pieces back together.
\item If one core is significantly longer than the other, break the longer core slightly to ensure both cores are approximately the same length.
\item Use white glue to glue the cores to the mounts, make sure the vertical vessels are perpendicular to the mount.
\item Use masking tape to secure the cores to the mounts.
\item After finishing one mount, use tape to tie the two cores from the same individual together.
	\end{enumerate}
\end{document}