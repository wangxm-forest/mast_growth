\documentclass[11pt]{article}

\usepackage{amsmath}
\usepackage{amssymb}
\usepackage{geometry}
\geometry{margin=1in}

\title{Growth-Reproduction Trade-off Model}
\author{Xiaomao Wang}
\date{}

\begin{document}
\maketitle

\section{Conceptual Framework}

We can approach the growth-reproduction trade-off from either the individual-tree level or the stand (population) level. Each approach has advantages and limitations, which will be discussed in the following subsections.

The overall idea is to construct a generative model in which trees (or stands) have an annual energy (carbon) pool that is allocated among different life processes. In this study, we focus on two processes only: reproduction and radial growth, so we will simplify the model into these two main components.

\subsection{Individual Level}

For each individual tree, we assume there is an energy (or carbon) pool available in year $t$. This allocation framework can be expressed as:
\begin{equation}
A_t = G_t + R_t,
\end{equation}
where:
\begin{itemize}
    \item $A_t$ is the energy (or carbon) assimilated in year $t$,
    \item $G_t$ is growth in year $t$,
    \item $R_t$ is reproduction in year $t$.
\end{itemize}

Ideally, reproduction should include both seed production and the formation of reproductive buds for the following year. However, since only seed production data are available, $R_t$ is restricted to seed production in this formulation.

\paragraph{Carbon Availability}

To quantify $A_t$, we could consider existing approaches such as the 3-PG model. However, a big challenge at the individual level is that most existing models for estimating NPP are at the stand level rather than the individual-tree level.

Alternatively, maybe we could introduce a latent variable representing the total carbon available for allocation in year $t$. We assume that carbon availability is driven by climate conditions (good years provide more resources) and tree size (larger trees may assimilate more carbon, potentially with an asymptote):
\begin{equation}
C_t \sim f(\text{climate}_t, \text{size}_t, \varepsilon),
\end{equation}
where $\varepsilon$ represents unexplained variation.

\paragraph{Carbon Allocation}

Carbon available in year $t$ is partitioned between reproduction and growth. For individual $i$ in year $t$, we define:
\begin{align}
R_{i,t} &= p_{i,t} \cdot C_{i,t}, \\
G_{i,t} &= (1 - p_{i,t}) \cdot C_{i,t},
\end{align}
where $p_{i,t}$ is the proportion of carbon allocated to reproduction.

\subsection{Stand Level}

At the stand level, NPP can be estimated directly. The only challenge here will be how to quantify the stand level growth. We could ever use the stand-level average or sum it up.
\paragraph{Carbon Availability}

Carbon availability at the stand level can be estimated using the 3-PG model. Required input data include:
\begin{itemize}
    \item Climate variables: solar radiation, temperature, rainfall, and vapor pressure deficit (VPD),
    \item Stand data: species composition, initial stocking (stems ha$^{-1}$), stand age, and basal area.
\end{itemize}

\paragraph{Carbon Allocation}

Similar to the individual-level formulation, carbon at the stand level is partitioned between reproduction and growth, either as a stand mean or a stand total. For stand $k$ in year $t$, we define:
\begin{align}
R_{k,t} &= p_{k,t} \cdot C_{k,t}, \\
G_{k,t} &= (1 - p_{k,t}) \cdot C_{k,t}.
\end{align}

Stand-level growth can also be expressed as the average of individual-level growth:
\begin{equation}
G_{k,t} = \frac{1}{N_{k,t}} \sum_{i=1}^{N_{k,t}} G_{i,t},
\end{equation}
where $N_{k,t}$ is the number of sampled trees in stand $k$ during year $t$.

\subsection{Pros and Cons}

Individual-level model allowa for finer resolution and better use of individual-level data when such data are available. However, estimating individual-level carbon assimilation requires strong assumptions and is might be challenging due to data limitations.

In the Mt. Rainier project, seed production is measured at the stand level using seed traps. Thus, using an individual-level model requires downscaling stand-level seed data to individual trees. Given known spatial coordinates for each tree and seed trap, it may be possible to estimate each tree’s contribution to seed production as a function of distance to the trap and tree size.

\end{document}
